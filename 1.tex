\documentclass[co342]{subfiles}

%% ========================================================
%% document

\begin{document}

    \chap{Connectivity}

    \section{Basic Graph Terminology}
    
    \begin{definition}{Graph}{}
        A \emph{graph} $G = \left( V,E \right)$ is an ordered pair such that
        \begin{enumerate}
            \item $V$ is a set; and\hfill\textit{vertex set}
            \item $E$ is a set of unordered pairs of elements of $V$.\hfill\textit{edge set}
        \end{enumerate}
        The elements of $V$ are called the \emph{vertices} of $G$ and the elements of $E$ are called the \emph{edges} of $G$. Given $u,v\in V$, we denote $uv$ to denote an edge \emph{joining} $u,v$. Note that $uv=vu$ for every $u,v\in V$. For simplicity, we assume that
        \begin{enumerate}
            \item $G$ does not have any \emph{loop} (i.e. $vv\notin E$ for every $v\in V$);\footnote{Note that, since we defined $E$ to be a set of unordered pairs of elements of $V$, if $e,f\in E$ join the same vertices, then $e=f$. In other words, we do not allow \emph{multiple edges}.} 
            \item $G$ has at least one vertex; and
            \item $G$ is \emph{finite} (i.e. $V$ is a finite set).
        \end{enumerate}
    \end{definition}

    \begin{notation}{$V\left( G \right) , E\left( G \right)$}{}
        Let $G$ be a graph. We write $V\left( G \right)$ to denote the vertex set and write $E\left( G \right)$ to denote the edge set of $G$.
    \end{notation}

    \begin{definition}{Subgraph}{of a Graph}
        Let $G$ be a graph. We say $H$ is a \emph{subgraph} of $G$ if $V\left( H \right) \subseteq V\left( G \right), E\left( H \right) \subseteq E\left( G \right)$, and all edges in $E\left( H \right)$ have both endpoints in $V\left( H \right)$. In particular, 
        \begin{enumerate}
            \item when $V\left( H \right) = V\left( G \right)$, we say $H$ is a \emph{spanning} subgraph; and
            \item if there exists $S\subseteq V\left( G \right)$ such that $V\left( H \right) = S$ and that
                \begin{equation*}
                    E\left( H \right) = \left\lbrace e\in E\left( G \right) : \exists v,u\in S\left[ e=vu \right]  \right\rbrace ,
                \end{equation*}
                then we say $H$ is the subgraph \emph{induced} by $S$, denoted as $G\left[ S \right]$.
        \end{enumerate}
    \end{definition}

    \begin{definition}{Degree}{of a Vertex}
        Let $G$ be a graph. The \emph{degree} of $v\in V\left( G \right)$, denoted as $d_G\left( v \right)$, is the number of edges incident to $v$. Moreover, we write $\delta\left( G \right)$ to denote
        \begin{equation*}
            \delta\left( G \right) = \min\left\lbrace d_G\left( v \right) : v\in V\left( G \right)  \right\rbrace 
        \end{equation*}
        and write $\Delta\left( G \right)$ to denote
        \begin{equation*}
            \Delta\left( G \right) = \max\left\lbrace d_G\left( v \right) : v\in V\left( G \right)  \right\rbrace .
        \end{equation*}
    \end{definition}

    \begin{definition}{Connected}{Graph}
        Let $G$ be a graph. We say $G$ is \emph{connected} if for every $u,v\in V\left( G \right)$, there is a $u,v$-path in $G$. A \emph{component} of $G$ is a maximally connected subgraph of $G$.
    \end{definition}

    \clearpage
    \begin{prop}{}
        Let $G$ be a graph. The following are equivalent.
        \begin{enumerate}
            \item $G$ is connected.
            \item There exists $v\in V\left( G \right)$ such that for every $u\in V\left( G \right)$, there exists a $u,v$-path in $G$.
        \end{enumerate}
    \end{prop}

    \begin{notation}{$G-F$, $G-e$, $G-U$, $G-v$}{}
        Let $G$ be a graph. 
        \begin{enumerate}
            \item Given $F\subseteq E\left( G \right)$, we write $G-F$ to denote the graph such that $V\left( G-F \right) = V\left( G \right) , E\left( G-F \right) = E\left( G \right) \setminus F$. 
            \item In case $F=\left\lbrace e \right\rbrace$ for some $e\in E$, we simply write $G-e$ to denote $G-\left\lbrace e \right\rbrace$. 
            \item Similarly, given $U\subseteq V\left( G \right) $, we write $G-U$ to denote the graph such that $V\left( G-U \right) = V\left( G \right) \setminus U$ and that
                \begin{equation*}
                    E\left( G-U \right) = \left\lbrace e\in E\left( G \right) : \exists u,v\in V\left( G \right) \setminus U\left[ e=uv \right]  \right\rbrace .
                \end{equation*}
                In other words, we remove every vertex in $U$ from $G$ and every edge in $G$ that is incident to at least one vertex in $U$.
            \item In case $U=\left\lbrace v \right\rbrace$ for some $v\in E$, we simply write $G-v$ to denote $G-\left\lbrace v \right\rbrace$.
        \end{enumerate}
    \end{notation}

    \begin{definition}{Contraction}{of a Graph}
        Let $G$ be a graph. For any $e\in E\left( G \right)$, the $G$ \emph{contract} $e$, denoted as $G /e$, is defined by \textit{removing} $e$ from $G$ and \textit{identifying} $u,v$ as the same vertex.
    \end{definition}

    \section{Edge Connectivity}
    
    \begin{definition}{$k$-edge-connected}{Graph}
        We say a graph $G$ is \emph{$k$-edge-connected} if $G-F$ is connected for every $F\subseteq E\left( G \right)$ such that $\left| F \right| < k$. The \emph{edge connectivity} of $G$, denoted as $\kappa'\left( G \right)$, is the largest $k\in\N$ for which $G$ is $k$-edge-connected.
    \end{definition}

    \begin{definition}{Cut-edge}{of a Graph}
        Let $G$ be a graph. We say $e\in E\left( G \right)$ is a \emph{cut-edge} if $G-\left\lbrace e \right\rbrace$ is disconnected.
    \end{definition}

    \np Let $G$ be a graph.
    \begin{enumerate}
        \item $\kappa'\left( G \right) = 1$ if and only if $G$ has a cut-edge. Equivalently, $G$ is $2$-edge connected if and only if $G$ is connected with no cut-edges.
        \item If $G$ is disconnected, then we define $\kappa'\left( G \right) = 0$ and say $G$ is $0$-edge-connected.
        \item The \emph{trivial graph} $\left( \left\lbrace v \right\rbrace , \emptyset \right)$ \textit{cannot be disconneted}; we define its edge-connectivity to be $0$.
    \end{enumerate}

    \clearpage
    \begin{prop}{}
        Let $G$ be a graph. Then
        \begin{equation*}
            \kappa'\left( G \right) \leq \delta\left( G \right) .
        \end{equation*}
    \end{prop}

    \begin{proof}
        If $\delta\left( G \right) = 0$, then there is an isolated vertex $v\in V\left( G \right)$, which means $G$ is disconnected. So $\kappa'\left( G \right) = 0$, meaning $\delta\left( G \right) = 0 \leq 0 = \kappa'\left( G \right)$. Now assume $\delta\left( G \right) \geq 1$ and let $v\in V\left( G \right)$ be such that $d_G\left( v \right) = \delta\left( G \right) $. Let $F = \left\lbrace e\in E\left( G \right) : \exists u\in V\left( G \right) \left[ uv=e \right]  \right\rbrace$, the set of all edges incident with $v$. Then $G-F$ is disconnected, since $v$ is not adjacent to any other vertices in $G$, which in turn exist since $\delta\left( G \right) \geq 1$. It follows that $\kappa'\left( G \right) \leq \left| F \right| = \delta\left( G \right) $. 
    \end{proof}

    \begin{definition}{Cut}{Induced by a Subset}
        Let $G$ be a graph and let $S\subseteq V\left( G \right)$. We define the \emph{cut} induced by $S$, denoted as $\delta_G\left( S \right)$, to be the set of all edges with exactly one end in $S$:
        \begin{equation*}
            \delta_G\left( S \right) = \left\lbrace e\in E\left( G \right) : \exists v\in V\left( G \right) \setminus S, u\in S\left[ e=vu \right]  \right\rbrace .
        \end{equation*}
        We say $\delta_G\left( S \right)$ is \emph{nontrivial} if $S\neq\emptyset$ and $S\neq V\left( G \right)$.
    \end{definition}

    \begin{prop}{}
        Let $G$ be a graph. Then the following are equivalent.
        \begin{enumerate}
            \item $G$ is connected.
            \item Every nontrivial cut of $G$ is nonempty.
        \end{enumerate}
    \end{prop}

    \np Say we have a graph and a cut $C\subseteq E\left( G \right)$. Then it is easy to see that $G-C$ is disconnected when $C$ is nontrivial. It follows that the size of any nontrivial cut is an upper bound on $\kappa'\left( G \right) $.

    \begin{prop}{}
        Let $G$ be a graph and let $F\subseteq E\left( G \right)$. If $G-F$ is disconnected, then $F$ contains a nontrivial cut.
    \end{prop}

    \begin{proof}
        Suppose $G-F$ is disconnected and let $H$ be a component of $G-F$. Since $G-F$ is disconnected, $V\left( H \right) \neq\emptyset$ and $V\left( H \right) \neq V\left( G \right) $. Consider $\delta_G\left( V\left( H \right)  \right)$. Since $H$ is maximally connected in $G-F$, none of the edges in $\delta_G\left( V\left( h \right)  \right)$ can be in $G-F$. Hence, all edeges in $|delta_G\left( V\left( H \right)  \right)$ must be in $F$. Hence $F$ contains a nontrivial cut $\delta_G\left( V\left( H \right) \right)$, as required.
    \end{proof}

    \begin{cor}{}
        Let $G$ be a graph. Then $\kappa'\left( G \right)$ is the size of a minimum nontrivial cut.
    \end{cor}	

    \begin{definition}{Bond}{of a Graph}
        A \emph{bond} is a minimal nonempty cut.
    \end{definition}

    \begin{prop}{Characterization of a Bond}
        Let $F=\delta_G\left( S \right)$ be a cut in a connected graph $G$, where $S\subseteq V\left( G \right)$. Then the following are equivalent.
        \begin{enumerate}
            \item $F$ is a bond.
            \item $G-F$ has exactly 2 components.
        \end{enumerate}
    \end{prop}

    \clearpage
    \begin{proof}
        \begin{itemize}
            \item ($\implies$) Suppose $G-F$ has at least $3$ components. Assume, without loss of generality, that (at least) $2$ of the components are contained in $G\left[ S \right]$ and let $H$ be one of them. Then $\delta_G\left( V\left( H \right)  \right) \subset F$, so $F$ is not minimal and hence not a bond.
            \item Suppose $G-F$ has exactly $2$ components and suppose further $F$ is not a bond, for the sake of contradiction. Then there is a nonempty cut $\delta_G\left( S \right)$ that is a proper subset of $F$. Then the two components of $G_F$ are linked by some edge in $F$. It follows that $G-\delta_G\left( S \right)$ is connected, contradicting the fact that $\delta_G\left( S \right)$ is a cut. \qqedsym
        \end{itemize} 
    \end{proof}

    \begin{prop}{}
        Every cut is a disjoint union of bonds.
    \end{prop}

    \begin{definition}{Symmetric Difference}{of Two Sets}
        Let $S,T$ be sets. Then the \emph{symmetric difference} of $S,T$, denoted as $S\triangle T$, is defined as
        \begin{equation*}
            S\triangle T = \left( S\cup T \right) \setminus\left( S\cap T \right) .
        \end{equation*}
    \end{definition}

    \begin{lemma_inside}{}
        Let $G$ be a graph and let $S,T\subseteq V\left( G \right)$. Then
        \begin{equation*}
            \delta_G\left( S \right) \triangle \delta_G\left( T \right) = \delta_G\left( S\triangle T \right).
        \end{equation*}
    \end{lemma_inside}

    \begin{proof}[Proof of Proposition 1.6]
        We proceed inductively on the number of edges in $F$. If $\left| F \right| = 0$, then is a disjoint union of no bonds, so we are done. Now assume t hat $F$ is any nonempty cut (where the inductive hypothesis is that any cut with a smaller size than $F$ is a disjoint union of bonds). We may assume $F$ is not a bond. Then $F$ contains a nonepmty cut $F_1$ by the definition of a bond. Now define
        \begin{equation*}
            F_2 = F\setminus F_1,
        \end{equation*}
        where we note that $F\setminus F_1 = \left( F\cup F_1 \right) \setminus \left( F\cap F_1 \right) = F\triangle F_1$. Hence by Lemma 1.6.1, $F_2$ is a cut. Since $F_1,F_2$ are nonempty, both $F_1,F_2$ have fewer edges than $F$. It follows that $F_1,F_2$ are disjoint union of bonds by the inductive hypothesis. Thus $F=F_1\cup F_2$ is also a disjoint union of bonds, as required.
    \end{proof}

    \section{Vertex Connectivity}
    
    \begin{definition}{Separating Set, Cut-vertex}{of a Graph}
        Let $G$ be a graph.
        \begin{enumerate}
            \item If $S\subseteq V\left( G \right)$ is such that $G-S$ is disconnected, then we say $S$ is a \emph{separating set}.
            \item If $v\in V\left( G \right)$ is such that $G-v$ has more components than $G$, then we say $v$ is a \emph{cut-vertex}.
        \end{enumerate}
    \end{definition}

    \begin{definition}{Connectivity}{of a Graph}
        Let $G$ be a graph. 
        \begin{enumerate}
            \item We say $G$ is \emph{$k$-connected}, where $k\in\N\cup\left\lbrace 0 \right\rbrace$, if $G$ has at least $k+1$ vertices and does not have a separating set of size at most $k-1$.
            \item The \emph{connectivity} of $G$, denoted as $\kappa\left( G \right)$, is the largest $k\in\N\cup\left\lbrace 0 \right\rbrace$ for which $G$ is $k$-connected.
        \end{enumerate}
    \end{definition}

    \np
    \begin{enumerate}
        \item The complete graph $K_n$ ($n\in\N$) has no separating sets. Hence $\kappa\left( K_n \right) = n-1$. In particular, $K_n$ is the smallest $\left(n-1 \right)$-connected graph: any $\left( n-1 \right)$-connected graph has at least $n$ vertices.
        \item Let $G$ be a graph. If there exists a separating set of size $k\in\K$, then $\kappa\left( G \right) \leq k$.
    \end{enumerate}

    \begin{theorem}{Expansion Lemma}
        Let $G$ be a $k$-connected graph and let $G'$ be obtained from $G$ by adding a new vertex $v$ and adding at least $k$ edges joining $v$ to at least $k$ distinct vertices of $G$. Then $G'$ is $k$-connected.
    \end{theorem}

    \begin{proof}
        To show that $G'$ is $k$-connected, we show that there is no separating set of size $k-1$. To do so, let $X\subseteq V\left( G' \right)$ have $k-1$ vertices. We now have $2$ cases.
        \begin{itemize}
            \item \textit{Case 1. Suppose $v\in X$.} Then $G'-X = G-\left( X\setminus \left\lbrace v \right\rbrace  \right)$. Since $G$ is $k$-connected and $\left| X\setminus \left\lbrace v \right\rbrace  \right| = k-2$, so $G-\left( X\setminus \left\lbrace v \right\rbrace  \right)$ is connected. Hence $G'-X$ is connected, implying that $X$ is not a separating set.
            \item \textit{Case 2. Suppose $v\notin X$.} Then $X\subseteq V\left( G \right)$. Since $\left| X \right| = k-1$ and $G$ is $k$-connected, $G-X$ is connected. Moreover, $v$ has at least $k$ neighbors by definition, so at least one of them, say $w$, is not in $X$. Then given any $u\in V\left( G-X \right)$, note that there is a $u,w$-path in $G-X$ by the connectedness of $G-X$, so by the fact that $w,v$ are neighbors in $G'-X$, there is a $u,v$-path in $G'-X$. Hence $G'-X$ is connected and $X$ is not a separating set.
        \end{itemize} 
        Thus $G'$ is $k$-connected, as required.
    \end{proof}

    \begin{theorem}{Whitney's Theorem}
        Let $G$ be a graph. Then
        \begin{equation*}
            \kappa\left( G \right) \leq \kappa'\left( G \right) \leq \delta\left( G \right) .
        \end{equation*}
    \end{theorem}

    \begin{proof}
        We showed the second inequality in Proposition 1.2. We now prove $\kappa\left( G \right) \leq \kappa'\left( G \right)$. Note that we have the following two special cases.
        \begin{itemize}
            \item \textit{Case 1. Suppose $G$ is complete, say $G=K_n$ for some $n\in\N$.} Then $\kappa\left( G \right) = \kappa'\left( G \right) = n-1$.
            \item \textit{Case 2. Suppose $G$ is disconnected.} Then $\kappa\left( G \right) = \kappa'\left( G \right) = 0$.
        \end{itemize} 
        Hence, we may assume $G$ is connected and not complete. Let $F$ be a minimum nontrivial cut in $G$, so that $\left| F \right| = \kappa'\left( G \right)$. Then $F$ is a bond, since every minimum nontrivial cut is clearly minimal. So by Proposition 1.5, $G-F$ has exactly $2$ components, say $H_1,H_2$. We break into two cases.
        \begin{itemize}
            \item \textit{Case 3. Suppose there exists a vertex $v\in V\left( G \right)$ that is not incident to any edge in $F$.} Without loss of generality, say $v$ is in $H_1$. Let $S$ b e the set of all vertices in $H_1$ incident with some edge in $F$. Then $\left| S \right| \leq \left| F \right|$. Also, $S$ is a separating set, since it separates $v$ from all the vertices in $H_2$. So
                \begin{equation*}
                    \kappa\left( G \right) \leq \left| S \right| \leq \left| F \right| = \kappa'\left( G \right) ,
                \end{equation*}
                as required.
            \item \textit{Case 4. Suppose all vertices are incident with some edge in $F$.} Since $G$ is not complete, there exist two vertices $u,v$ that are not adjacent. Without loss of generality, say $u\in H_1$. Let $S$ be the set of all neighbors of $u$. Now observe that $v\notin S$, so $S$ is a separating set as it separates $u$ from $v$. We now count the size of $S$. To do so, we show that every vertex in $S$ corresponds to a distinct edge in $F$.
                \begin{itemize}
                    \item \textit{Case 4.1. Say $y\in H_1$.} Then $y$ is incident to some edge in $F$, say $e$, (since we assumed that all vertices are incident with some edge in $F$), and clearly $u$ is not incident to $e$, since both $y,u$ are in $H_1$.
                    \item \textit{Case 4.2. Say $y\in H_2$.} Then $uy\in F$.
                \end{itemize} 
                In other words, if $y\in H_1$, then $y$ corresponds to an edge in $F$ that is incident to $y$ but not incident to $u$. If $y\in H_2$, then $y$ corresponds to $uy$. And it is immediate from the definition that these correspondences are distinct.\footnote{That is, we just informally described an injection from $S$ to $F$.} It follows that $\left| S \right| \leq \left| F \right|$, which means
                \begin{equation*}
                    \kappa\left( G \right) \leq \left| S \right| \leq \left| F \right| = \kappa'\left( G \right) .
                \end{equation*}
        \end{itemize} 
        This completes the proof.
    \end{proof}

    \section{Menger's Theorem}

    \np So far we have looked at the connectivity of graphs via \textit{destructive} methods: that is, we remove, or \textit{destroy}, few edges or vertices and see what happens. We now change our perspective to \textit{constructive} methods: finding disjoint paths between vertices. We first do this in a \textit{local} point view (i.e. given two vertices, we ask: how well connected are they?). Later, we extend this to a \textit{global} description, using paths.

    \np Let us reiterate what we said in (1.4). Suppose that $x,y$ are two nonadjacent vertices of a graph $G$. There are two ways to measure connectivity between $x,y$.
    \begin{enumerate}
        \item How many vertices do we need to remove to disconnect $x$ from $y$? \hfill\textit{destructive}
        \item How many \textit{internally disjoint} paths are there from $x$ to $y$? \hfill\textit{constructive}
    \end{enumerate}
    Note that for (a) we want the \textit{minimum} number of such vertices, whereas for (b) we want the \textit{maximum} number of such paths. In fact, we can ask the following question: between (a), (b), which one should be larger? 

    \begin{subproof}[Answer-ish]
        Suppose there are $k$ paths from $x$ to $y$. In order to disconnect $x$ from $y$, we have to remove at least one vertex from each path. In other words, (a) is at least (b). 
    \end{subproof}
    
    \noindent Surprisingly, \textit{Menger's theorem} shows that this inequality always holds with equality. Let us first define notions that we need.
    
    \begin{definition}{$X,Y$-separating Set}{}
        Let $G$ be a graph and let $X,Y\subseteq V\left( G \right)$. We say $S\subseteq V\left( G \right)\setminus\left( X\cup Y \right)$ is an \emph{$X,Y$-separating set} if there is no path from any vertex in $X$ to any vertex in $Y$ in $G-S$. In case $X=\left\lbrace x \right\rbrace,Y=\left\lbrace y \right\rbrace$, we say $S$ is an \emph{$x,y$-separating set}.
    \end{definition}

    \begin{definition}{Internally Disjoint}{$x,y$-paths}
        Let $G$ be a graph and let $x,y\in V\left( G \right)$. We say $x,y$-paths are \emph{internally disjoint} (or \emph{i.d.}), if no two of them share any vertices other than $x,y$.
    \end{definition}

    \noindent Note that, from our previous observations, if $S$ is an $x,y$-separating set and $P$ is a set of internally disjoint $x,y$-paths, then $\left| S \right| \geq \left| P \right|$.

    \begin{theorem}{Menger's Theorem}
        Let $x,y$ to nonadjacent distinct vertices of a graph $G$. Then the minimum size of an $x,y$-separating set is equal to the maximum size of a set of internally disjoint $x,y$-paths.
    \end{theorem}

    \begin{notation}{$G /X$}{}
        Let $G$ be a graph and let $X\subseteq V\left( G \right)$. We denote $G /X$ to be the graph obtained from $G$ by removing all edges in $G\left[ X \right]$ and identifying all vertices in $X$ into one new vertex.
    \end{notation}

    \np[Wrong Approach to Manger's Theorem]Here is a \textit{wrong}, but useful, approach to the Menger's theorem. We proceed inductively on the number of edges of the graph. Let $S$ be a minimum $x,y$-separating set, and say $k=\left| S \right|$. Let $X,Y$ be the components containing $x,y$ in $G-S$, respectively. Then we \textit{shrink} $X$ by taking $G /X$, and get $k$ i.d. $x,y$-paths by induction. Similarly, we \textit{shrink} $Y$ by taking $G /Y$, and get $k$ i.d. $x,y$-paths by induction. Combining these paths results in $k$ i.d. $x,y$-paths in $G$.

    \np[Proof Outline of Manger's Theorem]Here is why (1.6) is wrong: the induction works only when $G /X, G /Y$ have \textit{fewer} edges than $G$. But if $X = \left\lbrace x \right\rbrace$ or $Y=\left\lbrace y \right\rbrace$, this is not the case, and the induction fails. Hence, much of the work of the upcoming proof is done to ensure that $X,Y$ has at least $2$ vertices, a vertex besides $x,y$, respectively. Here is an outline.

    \begin{subproof}[Proof Outline]
        We begin by assuming that a minimum $x,y$-separating set, say $S$, has size $k$. We desire to build $k$ i.d. $x,y$-paths by using induction on $\left| E\left( G \right)  \right|$. We break our proof into three cases.
        \begin{itemize}
            \item \textit{Case 1. Suppose every edge is incident with $x,y$.} This case is relatively easy.
        \end{itemize} 
        After we are done with Case 1, we may assume that there are vertices $u,v\in V\left( G \right) \setminus \left\lbrace x,y \right\rbrace$ that are adjacent. The vertices $u,v$ will be the extra vertices we need in $X,Y$. Let $e=uv$.
        \begin{itemize}
            \item \textit{Case 2. Suppose a minimum $x,y$-separating set has size $k$ in $G-e$.} Then by applying induction we obtain $k$ i.d. $x,y$-paths.
            \item \textit{Case 3. Suppose a minimum $x,y$-separating set $S$ has size at most $k-1$ in $G-e$.} This is arguably the hardest case. For Case 3, we first show that any minimum $x,y$-separating set indeed has size $k-1$ in $G-e$. Afterwards, we show that $S\cup \left\lbrace u \right\rbrace$ is a minimum $x,y$-separating set of size $k$ in $G /Y$, and since $G /Y$ has fewer edges than $G$, by applying induction we obtain $k$ i.d. $x,y$-paths. Similarly, we obtain $k$ i.d. $x,y$-paths in $G /X$. Combining these paths gives us $k$ i.d. $x,y$-paths in $G$. \qqqedsym
        \end{itemize} 
    \end{subproof}

    \begin{proof}[Proof of Menger's Theorem]
        We prove by inducting on the number of edges on $G$. When there are no edges, a minimum $x,y$-separating set has size $0$, and there are $0$ $x,y$-paths. We now assume $G$ has at least one edge, and that the theorem holds for any graph with less than $\left| E\left( G \right) \right|$ edges. Let $k$ be the size of a minimum $x,y$-separating set, where the goal is to find $k$ internally disjoint $x,y$-paths. We have few cases.
        \begin{itemize}
            \item \textit{Case 1. Suppose every edge of $G$ is incident with $x$ or $y$.} Let $S$ be the set of all vertices adjacent to both $x,y$. Then the only $x,y$-paths are those of length $2$ going through $S$, and they are internally disjoint. So there are $\left| S \right|$ internally disjoint $x,y$-paths. But $S$ is a $x,y$-separating set, so $\left| S \right|=k$, as required.
        \end{itemize} 
        We now assume there exists some edge $uv$ that is not incident to $x$ nor $y$. Let $H = G-uv$ and let $S$ be a minimum $x,y$-separating set in $H$.
        \begin{itemize}
            \item \textit{Case 2. Suppose $\left| S \right| = k$.} Since $H$ has fewer edges than $G$, by induction there exists a set of $k$ internally disjoint $x,y$-paths in $H$. These paths are also in $G$ since $H\subseteq G$, and the result follows.
            \item \textit{Case 3. Suppose $\left| S \right|< k$.} 
                \begin{itemize}
                    \item \textit{Claim 1. We claim that $S\cup \left\lbrace u \right\rbrace$ is an $x,y$-separating set in $G$.}

                        \begin{subproof}
                            Note $G-\left( S\cup \left\lbrace u \right\rbrace \right)=H-\left( S\cup \left\lbrace u \right\rbrace \right)$. Also, $S\cup \left\lbrace u \right\rbrace$ is an $x,y$-separating set in $H$, since $H$ is an $x,y$-separating set in $H$. It follows that $S\cup\left\lbrace u \right\rbrace$ is an $x,y$-separating set in $G$. 
                        \end{subproof}
                \end{itemize} 
                But any $x,y$-separating set in $G$ in $G$ has size at least $k$, so $\left| S\cup\left\lbrace u \right\rbrace \right|\geq k$. Therefore, $\left| S \right|\geq k-1$, which means $\left| S \right|= k-1$. 
                Now write
                \begin{equation*}
                    S = \left\lbrace v_i \right\rbrace^{k-1}_{i=1}
                \end{equation*}
                for convenience. Let $X,Y$ be the vertices of the components of $H-S$ containing $x,y$, respectively. Since $\left| S \right|= k-1<k$, $G-S$ does not separate $x,y$. So there is an $x,y$-path in $G-S$. This path does not exist in $H-S$, so $uv$ must be on this path. In fact, $e$ must join a vertex in $X$ with a vertex in $Y$. Without loss of generality, say $u\in X, v\in Y$. This means $X,Y$ has at least two vertices, $X\supseteq \left\lbrace x,u \right\rbrace, Y\supseteq\left\lbrace y,v \right\rbrace$. We now consider $G /Y$, and let $y^{*}$ be the new vertex.
                \begin{itemize}
                    \item \textit{Claim 2. Any $x,y^{*}$-separating set in $G /Y$ is an $x,y$ separating set in $G$.}

                        \begin{subproof}
                            Let $T$ be any $x,y^{*}$-separating set in $G /Y$. Then any $x,y$-path in $G$ that does not use $T$ corresponds to an $x,y^{*}$-path in $G /Y$ that does not use $T$, which does not exist. Therefore, $T$ is an $x,y$-separating set in $G$. 
                        \end{subproof}
                \end{itemize} 
                Claim 2 in particualr means any $x,y^{*}$-separating set in $G /Y$ has size at least $k$. But $S\cup \left\lbrace u \right\rbrace$ is an $x,y^{*}$-separating set in $G /Y$, with size $k$. Therefore, a minimum $x,y^{*}$-separating set in $G /Y$ has size exactly $k$. Since $Y$ has at least $2$ vertices, $G\left[ Y \right]$ has att least one edge, so $G /Y$ has fewer edges than $G$, so by induction there is a set of $k$ internally disjoint $x,y^{*}$-paths in $G /Y$. In particular, each path must use a distinct vertex in $S\sup \left\lbrace u \right\rbrace$. Now for each $i\in\left\lbrace 1,\ldots,k-1 \right\rbrace$, let $P_i$ be the parts of the paths from $x$ to $v_i$, and let $P_k$ be the part of the path from $x$ to $u$. We can make the same argument for $G /X$, and find $k$ paths: $Q_1,\ldots,Q_{k-1}$ from $v_1,\ldots,v_{k-1}$ to $y$, $Q_k$ from $v$ to $y$. Then,
                \begin{equation*}
                    \left\lbrace P_i+Q_i \right\rbrace^{k-1}_{i=1}\cup \left\lbrace P_k+uv+Q_k \right\rbrace
                \end{equation*}
                is a set of $k$ internally disjoint $x,y$-paths, as required. \qqedsym
        \end{itemize} 
    \end{proof}

    \np There are many variations and corollaries of Menger's theorem.

    \begin{cor}{Characterization of $k$-connectedness}
        Let $G$ be a graph. The following are equivalent.
        \begin{enumerate}
            \item $G$ is $k$-connected.
            \item There exist $k$ internally disjoint $x,y$-paths for any distinct $x,y\in V\left( G \right)$.
        \end{enumerate}
    \end{cor}	

    \begin{proof}
        \begin{itemize}
            \item (a)$\implies$(b): Let $x,y\in V\left( G \right)$. We break into $2$ cases.
                \begin{itemize}
                    \item \textit{Case 1. Suppose $x,y$ are not adjacent.} Since $G$ is $k$-connected, any $x,y$-separating set has size at least $k$, so by Menger's theorem, there exist at least $k$ internally disjoint $x,y$-paths in $G$.
                    \item \textit{Case 2. Suppose $x,y$ are adjacent.} Then $G-xy$ is $\left( k-1 \right)$-connected. So any $x,y$-separating set in $G-xy$ has size at least $k-1$. By Menger's theorem, there exist at least $k-1$ internally disjoint $x,y$-paths in $G-xy$. Together with the path $\left( \left\lbrace x,y \right\rbrace, \left\lbrace xy \right\rbrace \right)$, we have at least $k$ internally disjoint $x,y$-paths in $G$.
                \end{itemize} 
            \item (a)$\impliedby$(b): Let $X$ be any set of $k-1$ vertices and let $x,y\in V\left( G \right)\setminus X$. By assumption, there exist $k$ internally disjoint $x,y$-paths in $G$. So removing $k-1$ vertices leaves at least one such path: there exists an $x,y$-path in $G-X$ for any vertices $x,y$ in $G-X$. Hence $G-X$ is connected, which means $G$ is $k$-connected. \qqedsym
        \end{itemize} 
    \end{proof}

    \begin{definition}{Fan}{}
        Let $G$ be a graph. Let $v\in V\left( G \right)$ and let $X\subseteq V\left( G \right)\setminus \left\lbrace v \right\rbrace$. A \emph{$k$-fan} from $v$ to $X$ is a set of $k$ internally disjoint paths that start with $v$ and end with distinct vertices in $X$, and no internal vertices are in $X$.
    \end{definition}

    \begin{cor}{Fan Lemma}
        Let $G$ be a $k$-connected graph. Let $v\in V\left( G \right)$ and let $X\subseteq V\left( G \right)\setminus \left\lbrace v \right\rbrace$ where $\left| X \right|\geq k$. Then there exists a $k$-fan from $v$ to $X$.
    \end{cor}	

    \begin{proof}
        Let $H$ be the graph obtained from $G$ from adding a vertex $x^{*}$ and edges $xx^{*}$ for all $x\in X$. Then observe that $H$ is $k$-connected by the Expansion lemma (Theorem 1.7), so by Corollary 1.9.1, there are $k$ internally disjoint $v,x^{*}$-paths, say $P_1,\ldots,P_k$. Now for each $i\in\left\lbrace 1,\ldots,k \right\rbrace$, write $P_i$ as
        \begin{equation*}
            P_i = \left( \left\lbrace v,u_1,\ldots,u_{l_i},x^{*} \right\rbrace, \left\lbrace vu_1,u_1u_2,\ldots,u_{l-1}u_l,u_lx^{*} \right\rbrace \right)
        \end{equation*}
        and let $j_i\in\left\lbrace 1,\ldots,l_i \right\rbrace$ be the smallest index such that $u_{j_i}\in X$. It follows that $\left\lbrace Q_i \right\rbrace^{k}_{i=1}$ by
        \begin{equation*}
            Q_i = \left( \left\lbrace v,u_1,\ldots,u_{j_i} \right\rbrace, \left\lbrace vu_1,u_1u_2,\ldots,u_{j_i-1}u_{j_i} \right\rbrace \right)
        \end{equation*}
        for all $i\in\left\lbrace 1,\ldots,k \right\rbrace$ is a $k$-fan from $v$ to $X$.
    \end{proof}

    \begin{cor}{}
        Let $G$ be a $k$-connected graph with $k\geq 2$ and let $X\subseteq V\left( G \right)$ be such that $\left| X \right|=k$. Then there exists a cycle in $G$ with all vertices in $X$.
    \end{cor}	

    \begin{proof}
        We proceed inductively on $k$. Suppose $k=2$ and let $X= \left\lbrace u,v \right\rbrace$. Then by Corollary 1.9.1, there exist $2$ internally disjoint $u,v$-paths in $G$; they form a cycle containing $u,v$. Now assume $k\geq 3$ and let $X\subseteq V\left( G \right)$ with $\left| X \right|=k$. Then $G$ is also $\left( k-1 \right)$-connected. Choose $v\in X$. Then $X\setminus \left\lbrace v \right\rbrace$ has $k-1$ vertices, so there is a cycle in $G$ that has all vertices in $X\setminus \left\lbrace v \right\rbrace$. If $C$ also has $v$, then we are done. So assume $C$ does not have $v$. The $k-1$ vertices of $X\setminus \left\lbrace v \right\rbrace$ are on the cycle $C$, say they are in order $v_1,\ldots,v_{k-1}$. These $k-1$ vertices partitions $C$ into $k-1$ internally disjoint paths, or segments. We now have $2$ cases.
        \begin{itemize}
            \item \textit{Case 1. Suppose $C$ has at least $k$ vertices.} Since $G$ is $k$-connected and $\left| V\left( G \right) \right|\geq k$, there exists a $k$-fan from $v$ to $V\left( C \right)$ by the Fan lemma. Since there are $k-1$ segment on $C$, by the pigeonhole principle, there is a segment $S$ on $C$ that contains the endpoints of $2$ paths, say $P_1,P_2$, in the fan. Say $P_1,P_2$ end with vertices $w_1,w_2\in V\left( C \right)$, respectively. Let $P_3$ be the $w_1,w_2$-path contained in $S$. Then no internal vertex of $P_3$ is in $X$, so $C-P_3+P_1+P_2$ is a cycle that contains $X$.
            \item \textit{Case 2. Suppose $C$ has exactly $k-1$ vertices.} Then there is a $\left( k-1 \right)$-fan from $v$ to $V\left( C \right)$, ending in all vertices in $C$. By removing any edge from $C$ and adding $2$ paths ending at the endpoints of the edge, we obtain a cycle containing $X$. \qqedsym
        \end{itemize} 
    \end{proof}

    \begin{definition}{Disjoint}{$X,Y$-paths}
        Let $G$ be a graph and let $X,Y\subseteq V\left( G \right)$. We say paths in $G$ are \emph{disjoint} $X,Y$-paths if 
        \begin{enumerate}
            \item each join a vertex in $X$ with a vertex in $Y$;
            \item no oehter vertices are in $X$ nor $Y$; and
            \item no two paths share any vertex, including the endpoints.
        \end{enumerate}
    \end{definition}

    \noindent Note that we \textit{allow} $X,Y$ to have a nonempty intersection in Def'n 1.15 and 1.18, and an $X,Y$-separating set could include vertices in $X$ or $Y$. In fact, $X$ and $Y$ are $X,Y$-separating sets.

    \begin{theorem}{Menger's Theorem, Set Version}
        Let $G$ be a graph and let $X,Y\subseteq V\left( G \right)$. Then the minimum size of an $X,Y$-separating set is equal to the maximum size of a set of disjoint $X,Y$-paths.
    \end{theorem}

    \begin{proof}
        First observe that any size of $X,Y$-separating set must be at least the size of any set of disjoint $X,Y$-paths, since we need to remove at least one vertex from each path. It remains to show that there exist an $X,Y$-separating set and a set of disjoint $X,Y$-paths of the same size. To do so, we shall utilize Menger's theorem. Let $H$ be the graph obrtained from $G$ by adding $2$ new vertices, say $x^{*},y^{*}$, and adding edges $x^{*}x, y^{*}y$ for all $x\in X, y\in Y$. Since $x^{*},y^{*}$ are not adjacent, by Menger's theorem, any minimum $x^{*},y^{*}$-separating set $S$ has the same size as a maximum set of internally disjoint $x,y$-paths $\mP$.
        \begin{itemize}
            \item \textit{Claim 1. $S$ is an $X,Y$-separating set in $G$.}

                \begin{subproof}
                    Suppose $S$ is not an $X,Y$-separating set, for the sake of contradiction. Then there exists and $X,Y$-path $Q$ from $x\in X$ to $y\in Y$ in $G-S$. Then $\left( V\left( Q \right)\cup\left\lbrace x^{*},y^{*} \right\rbrace, E\left( Q \right)\cup\left\lbrace x^{*}x,y^{*}y \right\rbrace \right)$ is an $x^{*},y^{*}$-path in $H-S$, contradicting the fact that $S$ is an $x^{*},y^{*}$-separating set.
                \end{subproof}
        \end{itemize} 
        We now obtain a set of disjoint $X,Y$-paths $\mP'$ from $\mP$ as follows. Say a path 
        \begin{equation*}
            \left( \left\lbrace x^{*},v_1,\ldots,v_l,y^{*} \right\rbrace, \left\lbrace x^{*}v_1,v_1v_2,\ldots,v_{l-1}v_l,v_ly^{*} \right\rbrace \right)\in\mP 
        \end{equation*}
        is given. Let $i\in\left\lbrace 1,\ldots,l \right\rbrace$ be the largest index such that $v_i\in X$, and let $j\in\left\lbrace 1,\ldots,l \right\rbrace$ be the smallest index at least $i$ such that $v_j\in Y$. We put the path $\left( \left\lbrace v_i,\ldots,v_j \right\rbrace, \left\lbrace v_iv_{i+1},\ldots,v_{j-1}v_j \right\rbrace \right)$ in $\mP'$. We see that each path in $\mP'$ joins a vertex in $X$ with a vertex in $Y$ with no other vertices in $X$ or $Y$, and paths in $\mP'$ do not share any vertices since $\mP$ is internally disjoint. Then $\left| S \right|=\left| \mP \right|=\left| \mP' \right|$, we have a set of disjoint $X,Y$-paths $\mP'$ of the same size as an $X,Y$-separating set $S$.
    \end{proof}

    \begin{definition}{Edge-disjoint}{$x,y$-paths}
        Let $G$ be a graph and let $x,y\in V\left( G \right)$. We say $x,y$-paths are \emph{edge-disjoint} if no $2$ paths share an edge.
    \end{definition}

    \begin{theorem}{Menger's Theorem, Edge Version}
        Let $x,y$ be distinct vertices of a graph $G$. Then the minimum size of a cut that disconnects $x$ from $y$ is equal to the maximum size of a set of edge-disjoint $x,y$-paths.
    \end{theorem}

    \begin{definition}{Line Graph}{of a Graph}
        Let $G$ be a graph. Then the \emph{line graph} of $G$, denoted as $\lgraph\left( G \right)$, is the graph whose vertex set is $\left\lbrace v_e:e\in E\left( G \right) \right\rbrace$ and $v_ev_f$ is an edge of $\lgraph\left( G \right)$ if and only if $e,f$ have a common endpoint in $G$.
    \end{definition}

    \clearpage
    \begin{cor}{Characterization of $k$-edge-connectedness}
        Let $G$ be a graph. The following are equivalent.
        \begin{enumerate}
            \item $G$ is $k$-edge-connected.
            \item There exist $k$ edge-disjoint $x,y$-paths for any distinct $x,y\in V\left( G \right)$.
        \end{enumerate}
    \end{cor}	

    \section{2-connected Graphs}
    
    \begin{prop}{Cycle Characterizations of 2-connectedness}
        Let $G$ be a graph with at least $3$ vertices. The following are equivalent.
        \begin{enumerate}
            \item $G$ is $2$-connected.
            \item For any distinct $u,v\in V\left( G \right)$, there is a cycle $C$ such that $u,v\in V\left( C \right)$.
            \item $\delta\left( G \right)\geq 1$ and for any distinct $e,f\in E\left( G \right)$, there is a cycle $C$ such that $e,f\in E\left( C \right)$.
        \end{enumerate}
    \end{prop}

    \begin{proof}
        \begin{itemize}
            \item (b)$\implies$(a) This follows immediately from Corollary 1.9.1.
            \item (a)$\implies$(c) Assume $G$ is $2$-connected. Since $G$ is connected with at least $3$ vertices, no isolated vertex exists, so $\delta\left( G \right)\geq 1$. Moreover, let $e,f\in E\left( G \right)$ be distinct and let $u,v,x,y\in V\left( G \right)$ be such that $e=uv, f=xy$. Let $A=\left\lbrace u,v \right\rbrace, B=\left\lbrace x,y \right\rbrace$. Since $G$ is $2$-connected, any $A,B$-separating set has size at least $2$. Hence by the set version of Menger's theorem (Theorem 1.10), there exist $2$ disjoint $A,B$-paths, $P_1,P_2$. In particular, both endpoints of $e,f$ are on distinct paths. So
                \begin{equation*}
                    C = P_1+f+P_2+e
                \end{equation*}
                is a cycle which has $e,f$.
            \item (c)$\implies$(b) Let $u,v\in V\left( G \right)$ be distinct. Since $\delta\left( G \right)\geq 1$, both $u,v$ are incident to at least one edge.
                \begin{itemize}
                    \item \textit{Claim 1. There exist distinct edges $e,f$ that are incident to $u,v$, respectively.}

                        \begin{subproof}
                            For the sake of contradiction, suppose $e\in E\left( G \right)$ is the only edge incident to $u,v$. Then $e=uv$. But this means
                            \begin{equation*}
                                H =\left( \left\lbrace u,v \right\rbrace, \left\lbrace uv \right\rbrace \right)
                            \end{equation*}
                            is a component of $G$, since no edge other than $e=uv$ is incident to $u,v$. Since $G$ has at least $3$ vertices, there is $w\in V\left( G \right)\setminus \left\lbrace v,u \right\rbrace$. But then there is $f\in E\left( G \right)$ that is incident to $w$, and $f\neq e$. Since $H$ is a component, it follows that there is no cycle of $G$ that has $e,f$, contradicting (c). 
                        \end{subproof}
                \end{itemize} 
                By Claim 1, choose distinct $e,f$ that are incident to $u,v$, respectively. Then by (c), there exists a cycle $C$ which has $e,f$, which means $C$ also has $u,v$.\qqedsym
        \end{itemize} 
    \end{proof}

    \begin{definition}{Ear Decomposition}{of a Graph}
        Let $F$ be a subgraph of $G$.
        \begin{enumerate}
            \item An \emph{ear} of $F$ in $G$ is a nontrivial path in $G$ where only the two endpoints of the path are in $F$.
            \item An \emph{ear decomposition} of $G$ is a sequence of graphs $\left( G_{0} \right)^{k}_{i=0}$ where
                \begin{enumerate}
                    \item $G_0$ is a cycle in $G$;\hfill\textit{starting cycle}
                    \item for each $r\in\left\lbrace 0,\ldots,k-1 \right\rbrace$,
                        \begin{equation*}
                            G_{i+1} = G_i+P_i
                        \end{equation*}
                        where $P_i$ is an ear of $G_i$; and\hfill\textit{intermediate cycles}
                    \item $G_k=G$.
                \end{enumerate}
        \end{enumerate}
    \end{definition}

    \begin{prop}{Characterization of 2-connectedness by Ear Decompositions}
        Let $G$ be a graph. The following are equivalent.
        \begin{enumerate}
            \item $G$ is $2$-connected.
            \item $G$ has an ear decomposition.
        \end{enumerate}
    \end{prop}

    \begin{proof}
        \begin{itemize}
            \item ($\implies$) Let $G$ be $2$-connected and let $G_0$ be any cycle in $G$. For $i\in\N\cup\left\lbrace 0 \right\rbrace$, we inductively construct $G_{i+1}$ from $G_i$ as follows.
                \begin{enumerate}
                    \item If $G_i=G$, then we are done.
                    \item Otherwise, there exists an edge $e\in E\left( G \right)\setminus E\left( G_i \right)$, say $e=uv$ for some $u,v\in V\left( G \right)$. Then by the set version of Menger's theorem, there exist $2$ disjoint $\left\lbrace u,v \right\rbrace,G_i$-path in $G$, say $Q_1,Q_2$. Then observe that
                        \begin{equation*}
                            P_i = Q_1+uv+Q_2
                        \end{equation*}
                        is an ear of $G_i$. We then set $G_{i+1}=G_i+P_i$.
                \end{enumerate}
                Note that the above process terminates since we add at least one edge of $G$ each time. At termination, say after $k$ step, $\left( G_{i} \right)^{k}_{i=1}$ is an ear decomposition of $G$.
            \item ($\impliedby$) Suppose $\left( G_{i} \right)^{k}_{i=0}$ is an ear decomposition of $G$. We proceed inductively to prove each $G_i$ is $2$-connected, thereby proving that $G$ is also $2$-connected. Since $G_0$ is a cycle, $G_0$ is $2$-connected. Now suppose that $G_i$ is $2$-connected, where $i\in\N\cup\left\lbrace 0 \right\rbrace$ and let $P_i$ be an ear of $G_i$ such that
                \begin{equation*}
                    G_{i+1} = G_i+P_i.
                \end{equation*}
                We claim that every distinct vertices $x,y$ in $G_{i+1}$ are in a cycle, so that we can use Proposition 1.12. Suppose that the two endpoins of $P_i$ are $u,v$ and let $x,y\in V\left( G_{i+1} \right)$. We have $3$ cases.
                \begin{itemize}
                    \item \textit{Case 1. Suppose $x,y\in V\left( G_i \right)$.} Then since $G_i$ is $2$-connected, there is a cycle $C$ in $G_i$, hence in $G_{i+1}$, which has $x,y$.
                    \item \textit{Case 2. Suppose $x,y\in V\left( P_i \right)$.} Since $G_i$ is connected, there exists a $u,v$-path $Q$ in $G_i$. It follows that $P_i+Q$ is a cycle which has $x,y$.
                    \item \textit{Case 3. Suppose exactly one of $x,y$ is in $G_i$.} Without loss of generality, say $x\in V\left( G_i \right), y\in V\left( P_i \right)$. Since $G_i$ is $2$-connected, there exist $2$-fan from $x$ to $\left\lbrace u,v \right\rbrace$ by the fan lemma. Then $2$-fan combined with $P_i$ is a cycle which has $x,y$.
                \end{itemize} 
                Hence every $2$ distinct vertices in $G_{i+1}$ are in a cycle, and by Proposition 1.12 $G_{i+1}$ is $2$-connected. Thus $G_k$ is $2$-connected, as required. \qqedsym
        \end{itemize} 
    \end{proof}

    \clearpage
    \begin{cor}{}
        Let $G$ be a graph.
        \begin{enumerate}
            \item If $G$ is $2$-connected, then any cycle in $G$ can be the starting cycle in the ear decomposition.
            \item If $G$ admits an ear decomposition, then each intermediate graph in the ear deecomposition is $2$-connected.
        \end{enumerate}
    \end{cor}

    \begin{prop}{}
        Let $G$ be a graph. If $G$ admits an ear decomposition $\left( G_{0} \right)^{k}_{i=0}$, then $k=\left| E\left( G \right) \right|-\left| V\left( G \right) \right|$.
    \end{prop}


    
    









































\end{document}
